% LaTeX resume using res.cls
\documentclass[margin]{res}
%\usepackage{helvetica} % uses helvetica postscript font (download helvetica.sty)
%\usepackage{newcent}   % uses new century schoolbook postscript font 
\setlength{\sectionwidth}{1.5in} % set width of section portion
\setlength{\resumewidth}{7.1in}  % set width of resume portion
\usepackage{enumitem}
\usepackage[scaled=0.85]{berasans}
%\usepackage{DejaVuSans}
%\usepackage{DejaVuSansCondensed}
%\usepackage{dejavu}
%\usepackage[default]{sourcesanspro}
%\usepackage{helvet}
\renewcommand{\familydefault}{\sfdefault}
\usepackage[T1]{fontenc}

\begin{document}
\voffset-0.5in
% Center the name over the entire width of resume:
 \moveleft.5\hoffset\centerline{\large\bf Ali Kakakhel}
% Draw a horizontal line the whole width of resume:
 \moveleft1.33\hoffset\vbox{\hrule width\resumewidth height 1pt}%\smallskip
% address begins here
% Again, the address lines must be centered over entire width of resume:
 % \moveleft.5\hoffset\centerline{2104 \ \ \  59th \ \ \  Ave \ \ \  Ct}
 % \moveleft.5\hoffset\centerline{Greeley, \ \  CO \ \  80634}
 % \moveleft.5\hoffset\centerline{alikakakhel@yahoo.com}
 % \moveleft.5\hoffset\centerline{(970) 396-5137}
\vskip-0.15in
2104 \ \ \  59th \ \ \  Ave \ \ \  Ct \qquad\qquad\ \ \ alikakakhel@yahoo.com \\
Greeley, \ \  CO \ \  80634 \qquad\qquad\qquad\qquad\qquad(970) 396-5137

\begin{resume}
 
%\section{OBJECTIVE}  A position in the field of computers with special 
%                interests in business applications programming, 
%                information processing, and management systems. 
 

\section{Education}
{\sl Masters of Science,} Radiological Physics \hfill May 2012\\
Wayne State University, Detroit, MI

{\sl Bachelor of Science,} Physics \hfill May 2008\\
Colorado State University, Fort Collins, CO
Minor: Mathematics
 
\section{Experience} 
{\sl Medical Physics Volunteer} \hfill May 2013 -- Present \\
North Colorado Medical Center, Greeley, CO
 \begin{itemize}  \itemsep -2pt %reduce space between items
    \item Commissioning TrueBeam for Pinnacle
    \item Tabulating beam data for TrueBeam
    \item HDR source swapping and daily quality assurance
    \item Assisting with OBI, linac, and patient specific delivery quality assurance
    \item Implementing a system to quantify and improve treatment plan quality
    \item Testing protocol for VisionRT gated simulation and treatment
    \item Molding electron cone cutouts
    \item 3D printing bolus and patient structures
\end{itemize}

{\sl Medical Physics Volunteer} \hfill Apr 2010 -- Apr 2011 \\
Karmanos Cancer Center, Detroit, MI
 \begin{itemize}  \itemsep -2pt %reduce space between items
    \item Observing and assisting with Gammaknife SRS treatments
    \item Commissioning new Teflon phantom for Eclipse
    \item Treatment planning: IMRT, SBRT, RapidArc, Tomotherapy
\end{itemize}

{\sl Tomotherapy QA Intern} \hfill         Feb 2010 -- Apr 2010 \\
Karmanos Cancer Center, Detroit, MI
 \begin{itemize}  \itemsep -2pt %reduce space between items
    \item Became familiar with Tomotherapy linac, software, and Cheese phantom
    \item Performed bi-weekly patient specific delivery quality assurance
 \end{itemize} 
{\sl IMRT QA Intern} \hfill                Dec 2009 -- Feb 2010 \\
Karmanos Cancer Center, Detroit, MI
  \begin{itemize}
    \item Became familiar with Mapcheck diode array and the PTW 729 ion \\ chamber array with Octavius phantom
    \item Performed bi-weekly patient specific delivery quality assurance
    \item Evaluated measurement results using VeriSoft verification software
   \end{itemize} 

\section{Certification} American Board of Radiology Therapetic Medical Physics Part 1

\section{Procedures}
\begin{itemize}[noitemsep]
    \item HDR source swapping and daily quality assurance
    \item Patient specific delivery quality assurance
    \item Monthly, annual linac quality assurance
    \item Tomotherapy quality assurance
    \item OBI quality assurance
    \item TrueBeam commissioning
    \item New phantom commissioning
    \item Radiation survey and risk assessment
    \item Radiation therapy vault shielding calculations
    \item Treatment planning: IMRT, SBRT, RapidArc, \\ Tomotherapy
    \item Electon cone cutout molding
    \item 3D printing bolus and patient structures
\end{itemize}

\section{Equipment}
\begin{itemize}[noitemsep]
    \item Varian TrueBeam, Trilogy IX, Clinac 2100 EX
    \item Varian GammaMed HDR Afterloader
    \item TomoTherapy Hi-ART II
    \item GammaKnife SRS
    \item 3D Scanning Water Tank
    \item MapCheck, ArcCheck, PTW 729
    \item Cheese, Octavius, Lucy Phantoms
    \item Ion Chamber, Film, Diode, TLD/OSLD
    \item MakerBot Replicator 2X
\end{itemize}

\section{Software}
\begin{itemize}[noitemsep]
    \item Eclipse, Tomotherapy, Mosaiq, Pinnacle
    \item RIT, DoseLab, VeriSoft, MapCheck, RadCalc, 3DVH, Quality Reports, OmniPro
    \item Matlab/Octave, R, Python, C/C++, LaTeX, XML
    \item 3DSlicer, Blender, Meshlab, Slic3r
\end{itemize}

\section{Research}
\begin{itemize} \itemsep -2pt % Reduce space between items
    \item Lack, D., Kakakhel, A., Starin, R., \& Snyder, M. (2014). \textit{Teflon Cylindrical Phantom for Delivery Quality Assurance of Stereotactic Body Radiotherapy (SBRT)}. Journal Of Applied Clinical Medical Physics, 15(1). \\ doi:10.1120/jacmp.v15i1.4536
    \item Kakakhel, A., Snyder, M., Lack, D. (2011). \textit{Applicability of Image Smoothing for Dose Calculation of High Density Phantoms in Patient Specific Delivery Quality Assurance}. Poster session presented at 2011 Joint AAPM/COMP Meeting, Vancouver, BC, Canada. \\ http://www.aapm.org/meetings/2011AM/PRAbs.asp?aid=15361
    \item Kakakhel, A. 2011. \textit{Commissioning a Cylindrical Teflon Phantom for Delivery Quality Assurance of SBRT Cases} (Master's Thesis). Wayne State University, Detroit, MI.
\end{itemize}
 
% \section{Honors}
% \begin{description}[noitemsep]
%     \item[2007 -- 2008]{Society of Physics Students Vice-President}
%     \item[2007]{Induction to Sigma Pi Sigma Physics Honor Society}
%     \item[2006 -- 2007]{Dean's List}
% \end{description}
 

\end{resume}
\end{document}




